\section{Virtual Private Cloud}

\begin{frame}
	\frametitle{Virtual Private Cloud (VPC)}
	\begin{itemize}
		\item VPCs são isoladas entre si, mas podem ser configuradas para se comunicarem
		\item Cada região tem uma VPC padrão, mas é recomendada criar sua própria VPC para o ambiente de produção
		\item Dentro de uma VPC é possível criar uma subnet
		\item As subnets são aplicadas em AZs (Zonas de disponibilidade)
		\item Subnet:
			\begin{itemize}
				\item Pública: Pode ser acessada remotamente por qualquer lugar
				\item Privada: Só vai ser acessível por dentro da AWS
			\end{itemize}
		\item \textbf{VPC wizard} tem algumas configurações pré-definidas de VPC
		\item Lembrar de verificar e configurar:
		\begin{itemize}
			\item DHCP options set
			\item DNS resolution
			\item DNS hostname
		\end{itemize}
	\end{itemize}
\end{frame}

\begin{frame}
	\frametitle{Internet Gateway}
	\begin{itemize}
		\item Libera a entrada e a saída de determinado \textbf{Route Table}
		\item Não tem custo
	\end{itemize}
\end{frame}

\begin{frame}
	\frametitle{Route table}
	\begin{itemize}
		\item Associa as \textbf{subnets}
		\item Se a \textbf{Route table} não tiver uma rota default ela não está pública
	\end{itemize}
\end{frame}

\begin{frame}
	\frametitle{Security Groups X NetworkACL}
	\begin{itemize}
		\item \textbf{Securty Groups}
		\begin{itemize}
			\item Opera no nível de instância (Primeira camada de defesa)
			\item Apenas regras de liberação
			\item Stateful: o tráfego de retorno é automaticamente permitido, independentemente de quaisquer regras
			\item Aplica-se a uma instância somente quando especificado o grupo de segurança
		\end{itemize}
		\item \textbf{NetworkACL}
		\begin{itemize}
			\item Regra de segurança da rede (Como se fosse um \textit{firewall})
			\item Regras de liberação e negação
			\item Stateless: o tráfego de retorno deve ser explicitamente permitido pelas regras
			\item Aplica a todas as instâncias nas sub-redes
		\end{itemize}
	\end{itemize}
\end{frame}

\begin{frame}
	\frametitle{NetworkACL}
	\begin{itemize}
		\item Cada regra vai ter uma prioridade
		\item É bom deixar um espaço entre cada regra para possíveis regras futuras (Ex: deixar 10 espaços entre cada regra)
		\item \textbf{OBS}: É bom liberar portas efêmeras (1024-65535). São usadas para comunicações de saída através do protocolo de rede TCP/IP
	\end{itemize}
\end{frame}
