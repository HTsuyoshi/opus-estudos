\section{Cloud VS}

\subsection{Cloud Vs Virtualização}

\begin{frame}
	\frametitle{Virtualização}
	\begin{itemize}
		\item Criação de infraestruturas virtuais a partir de uma estrutura física
	\end{itemize}
\end{frame}

\begin{frame}
	\frametitle{Cloud}
	\begin{itemize}
		\item Conceito que reúne vários \textit{softwares} e utiliza de virtualização
		\item Possui algumas características específicas:
			\begin{itemize}
				\item Autoserviço sob demanda
				\item Amplo acesso a rede
				\item Pool de recursos
				\item Rápida elaticidade
				\item Serviços mensuráveis
			\end{itemize}
	\end{itemize}
\end{frame}

\subsection{Cloud Vs Virtual hostings/Virtual private server}

\begin{frame}
	\frametitle{Cloud vs Virtual Hosting/Virtual Private Server}
	\begin{itemize}
		\item VH ou VPS não podem ser considerados computação em nuvem
		\item \textbf{Hosting}: O contratante fica responsável apenas pela camada da aplicação
		\item \textbf{Colocation}: 1000 \textbf{VMs} na rede, \textbf{VPS} por um provedor, servidor físico em um provedor, etc\dots
	\end{itemize}
\end{frame}

