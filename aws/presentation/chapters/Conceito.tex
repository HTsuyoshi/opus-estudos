\section{Conceito}

\subsection{NIST}

\begin{frame}
	\frametitle{O que é cloud para o NIST}
	\begin{itemize}
		\item Um modelo para habilitar o acesso por rede a um conjunto compartilhado de recursos de computação e precisa ser:
			\begin{itemize}
				\item Ubíquo (Pode ser encontrado em todos os lugares)
				\item Conveniente
				\item Sob demanda
			\end{itemize}
		\item \textbf{Recursos de computação}: Redes, servidores, armazenamento, aplicações e serviços
		\item Esses recursos devem ser provisionados e liberados com o mínimo de esforço de gerenciamento ou interação com o provedor de serviços.
	\end{itemize}
\end{frame}

\subsection{5 características}

\begin{frame}
	\frametitle{5 Características}
	\begin{itemize}
		\item Conceito que reúne vários \textit{softwares} e utiliza de virtualização
		\item Possui algumas características específicas (NIST):
			\begin{itemize}
				\item Autoserviço sob demanda
				\item Amplo acesso a rede
				\item Pool de recursos
				\item Rápida elaticidade
				\item Serviços mensuráveis
			\end{itemize}
	\end{itemize}
\end{frame}

\begin{frame}
	\frametitle{Auto-serviço sob demanda}
	\begin{itemize}
		\item O consumidor pode providionar por conta própra \textbf{Recursos de computação}
		\item Não necessita da intervenção humana dos provedores de serviço
	\end{itemize}
\end{frame}

\begin{frame}
	\frametitle{Amplo acesso por rede}
	\begin{itemize}
		\item Os \textbf{Recursos de computação} estão disponíveis através da rede
		\item São acessados através de mecanismos padronizados que promovem o uso por dispositivos, clientes leves ou ricos de diversas plataformas (Smartphones, tablets, laptops ou desktops)
	\end{itemize}
\end{frame}

\begin{frame}
	\frametitle{Agrupamento de recursos}
	\begin{itemize}
		\item Os \textbf{Recusos de computação} do provedor são agrupados para atender a múltiplos consumidores em modalidade multi-inquilinos (Recursos físicos e virtuais diferentes dinamicamentes atribuídos e reatribuídos conforme a demanda dos consumidores)
		\item Há uma certa independência de localização geográfica, uma vez que o consumidor em geral não controla ou conhece a localização exata dos recursos fornecidos
		\item Mas pode ser capaz de especificar a localização em um nível de abstração mais alto (país, estado, datacenter)
	\end{itemize}
\end{frame}

\begin{frame}
	\frametitle{Elasticidade rápida}
	\begin{itemize}
		\item Os recursos podem ser provisionados e liberados elasticamente, em alguns casos automaticamentes, para rapidamente aumentar ou diminuir de acordo com a demanda
		\item Para o consumidor, os recursos disponíveis para provisionamento muitas vezes parecem ser ilimitados e podem ser alocados em qualquer quantidade e a qualquer tempo
	\end{itemize}
\end{frame}

\begin{frame}
	\frametitle{Serviços mensurado}
	\begin{itemize}
		\item Os sistemas na nuvem automaticamente controlam e otimizam o uso dos recursos através de medições em um nível de abstração apropriado para o tipo de serviço (como armazenamento, processamento, comunicação de ree e contas de usuário ativas)
		\item A utilização de recursos pode ser monitorada, controlada e informada, gerando transparência tanto para o fornecedor como para o consumidor do serviço utilizado
	\end{itemize}
\end{frame}
