\section{Conceito}

\subsection{Cloud X Virtualização}

\begin{frame}
	\frametitle{Virtualização}
	\begin{itemize}
		\item Criação de infraestruturas virtuais a partir de uma estrutura física
	\end{itemize}
\end{frame}

\begin{frame}
	\frametitle{Cloud}
	\begin{itemize}
		\item Conceito que reúne vários \textit{softwares} e utiliza de virtualização
		\item Possui algumas características específicas:
			\begin{itemize}
				\item Autoserviço sob demanda
				\item Amplo acesso a rede
				\item Pool de recursos
				\item Rápida elaticidade
				\item Serviços mensuráveis
			\end{itemize}
		\item \textbf{Colocation}: alugar determinado espaço de um Data Center (refrigeração, instalação elétrica e segurança). No entanto, utiliza seu próprio equipamento. \footnote{\href{https://vivomeunegocio.com.br/conteudos-gerais/expandir/cloud-colocation-data-center}{Colocation}}
	\end{itemize}
\end{frame}

\subsection{NIST}

\begin{frame}
	\frametitle{O que é cloud para o NIST}
	\begin{itemize}
		\item Um modelo para habilitar o acesso por rede a um conjunto compartilhado de \textbf{recursos de computação} e precisa ser:
			\begin{itemize}
				\item Ubíquo (Pode ser encontrado em todos os lugares)
				\item Conveniente
				\item Sob demanda
			\end{itemize}
		\item \textbf{Recursos de computação}: Redes, servidores, armazenamento, aplicações e serviços
		\item Esses recursos devem ser provisionados e liberados com o mínimo de esforço de gerenciamento ou interação com o provedor de serviços.
	\end{itemize}
\end{frame}

\begin{frame}
	\frametitle{5 Características}
	\begin{itemize}
		\item Conceito que reúne vários \textit{softwares} e utiliza de virtualização
		\item Possui algumas características específicas (NIST):
			\begin{itemize}
				\item Autoserviço sob demanda
				\item Amplo acesso a rede
				\item Pool de recursos
				\item Rápida elaticidade
				\item Serviços mensuráveis
			\end{itemize}
	\end{itemize}
\end{frame}

\begin{frame}
	\frametitle{Auto-serviço sob demanda}
	\begin{itemize}
		\item O consumidor pode providionar por conta própra \textbf{Recursos de computação}
		\item Não necessita da intervenção humana dos provedores de serviço
	\end{itemize}
\end{frame}

\begin{frame}
	\frametitle{Amplo acesso por rede}
	\begin{itemize}
		\item Os \textbf{Recursos de computação} estão disponíveis através da rede
		\item São acessados através de mecanismos padronizados que promovem o uso por dispositivos, clientes leves ou ricos de diversas plataformas (Smartphones, tablets, laptops ou desktops)
	\end{itemize}
\end{frame}

\begin{frame}
	\frametitle{Agrupamento de recursos}
	\begin{itemize}
		\item Os \textbf{Recusos de computação} do provedor são agrupados para atender a múltiplos consumidores em modalidade multi-inquilinos (Recursos físicos e virtuais diferentes dinamicamentes atribuídos e reatribuídos conforme a demanda dos consumidores)
		\item Há uma certa independência de localização geográfica, uma vez que o consumidor em geral não controla ou conhece a localização exata dos recursos fornecidos
		\item Mas pode ser capaz de especificar a localização em um nível de abstração mais alto (país, estado, datacenter)
	\end{itemize}
\end{frame}

\begin{frame}
	\frametitle{Elasticidade rápida}
	\begin{itemize}
		\item Os recursos podem ser provisionados e liberados elasticamente, em alguns casos automaticamentes, para rapidamente aumentar ou diminuir de acordo com a demanda
		\item Para o consumidor, os recursos disponíveis para provisionamento muitas vezes parecem ser ilimitados e podem ser alocados em qualquer quantidade e a qualquer tempo
	\end{itemize}
\end{frame}

\begin{frame}
	\frametitle{Serviços mensurado}
	\begin{itemize}
		\item Os sistemas na nuvem automaticamente controlam e otimizam o uso dos recursos através de medições em um nível de abstração apropriado para o tipo de serviço (como armazenamento, processamento, comunicação de ree e contas de usuário ativas)
		\item A utilização de recursos pode ser monitorada, controlada e informada, gerando transparência tanto para o fornecedor como para o consumidor do serviço utilizado
	\end{itemize}
\end{frame}

\begin{frame}
	\frametitle{Primeiros dias da computação em nuvem}
	\begin{figure}[htpb]
	\begin{center}
	\begin{tiny}
	\begin{tikzpicture}[scale=1, transform shape]
		\node (A) at (0,4) {\normalsize On-Site};
		\node (A1) at ($(A) - (0,0.6)$)  [draw,thick,minimum width=75,minimum height=15] {Applicação};
		\node (A2) at ($(A1) - (0,0.6)$) [draw,thick,minimum width=75,minimum height=15] {Data};
		\node (A3) at ($(A2) - (0,0.6)$) [draw,thick,minimum width=75,minimum height=15] {Runtime};
		\node (A4) at ($(A3) - (0,0.6)$) [draw,thick,minimum width=75,minimum height=15] {Middleware};
		\node (A5) at ($(A4) - (0,0.6)$) [draw,thick,minimum width=75,minimum height=15] {O/S};
		\node (A6) at ($(A5) - (0,0.6)$) [draw,thick,minimum width=75,minimum height=15] {Virtualization};
		\node (A7) at ($(A6) - (0,0.6)$) [draw,thick,minimum width=75,minimum height=15] {Servers};
		\node (A8) at ($(A7) - (0,0.6)$) [draw,thick,minimum width=75,minimum height=15] {Storage};
		\node (A9) at ($(A8) - (0,0.6)$) [draw,thick,minimum width=75,minimum height=15] {Networking};
		\node (B) at (3,4) {\normalsize IaaS};
		\node (B1) at ($(B) - (0,0.6)$)  [draw,thick,minimum width=75,minimum height=15] {Applicação};
		\node (B2) at ($(B1) - (0,0.6)$) [draw,thick,minimum width=75,minimum height=15] {Data};
		\node (B3) at ($(B2) - (0,0.6)$) [draw,thick,minimum width=75,minimum height=15] {Runtime};
		\node (B4) at ($(B3) - (0,0.6)$) [draw,thick,minimum width=75,minimum height=15] {Middleware};
		\node (B5) at ($(B4) - (0,0.6)$) [draw,thick,minimum width=75,minimum height=15] {O/S};
		\node (B6) at ($(B5) - (0,0.6)$) [lightgreen,draw,thick,minimum width=75,minimum height=15] {Virtualization};
		\node (B7) at ($(B6) - (0,0.6)$) [lightgreen,draw,thick,minimum width=75,minimum height=15] {Servers};
		\node (B8) at ($(B7) - (0,0.6)$) [lightgreen,draw,thick,minimum width=75,minimum height=15] {Storage};
		\node (B9) at ($(B8) - (0,0.6)$) [lightgreen,draw,thick,minimum width=75,minimum height=15] {Networking};
		\node (C) at (6,4) {\normalsize PaaS};
		\node (C1) at ($(C) - (0,0.6)$)  [draw,thick,minimum width=75,minimum height=15] {Applicação};
		\node (C2) at ($(C1) - (0,0.6)$) [draw,thick,minimum width=75,minimum height=15] {Data};
		\node (C3) at ($(C2) - (0,0.6)$) [lightgreen,draw,thick,minimum width=75,minimum height=15] {Runtime};
		\node (C4) at ($(C3) - (0,0.6)$) [lightgreen,draw,thick,minimum width=75,minimum height=15] {Middleware};
		\node (C5) at ($(C4) - (0,0.6)$) [lightgreen,draw,thick,minimum width=75,minimum height=15] {O/S};
		\node (C6) at ($(C5) - (0,0.6)$) [lightgreen,draw,thick,minimum width=75,minimum height=15] {Virtualization};
		\node (C7) at ($(C6) - (0,0.6)$) [lightgreen,draw,thick,minimum width=75,minimum height=15] {Servers};
		\node (C8) at ($(C7) - (0,0.6)$) [lightgreen,draw,thick,minimum width=75,minimum height=15] {Storage};
		\node (C9) at ($(C8) - (0,0.6)$) [lightgreen,draw,thick,minimum width=75,minimum height=15] {Networking};
		\node (D) at (9,4) {\normalsize SaaS};
		\node (D1) at ($(D) - (0,0.6)$)  [lightgreen,draw,thick,minimum width=75,minimum height=15] {Applicação};
		\node (D2) at ($(D1) - (0,0.6)$) [lightgreen,draw,thick,minimum width=75,minimum height=15] {Data};
		\node (D3) at ($(D2) - (0,0.6)$) [lightgreen,draw,thick,minimum width=75,minimum height=15] {Runtime};
		\node (D4) at ($(D3) - (0,0.6)$) [lightgreen,draw,thick,minimum width=75,minimum height=15] {Middleware};
		\node (D5) at ($(D4) - (0,0.6)$) [lightgreen,draw,thick,minimum width=75,minimum height=15] {O/S};
		\node (D6) at ($(D5) - (0,0.6)$) [lightgreen,draw,thick,minimum width=75,minimum height=15] {Virtualization};
		\node (D7) at ($(D6) - (0,0.6)$) [lightgreen,draw,thick,minimum width=75,minimum height=15] {Servers};
		\node (D8) at ($(D7) - (0,0.6)$) [lightgreen,draw,thick,minimum width=75,minimum height=15] {Storage};
		\node (D9) at ($(D8) - (0,0.6)$) [lightgreen,draw,thick,minimum width=75,minimum height=15] {Networking};
	\end{tikzpicture}
	\end{tiny}
	\end{center}
	\caption{Imagem retirado do site da redhat\footnote{\href{https://www.redhat.com/cms/managed-files/iaas-paas-saas-diagram5.1-1638x1046.png}{RedHat IaaS, PaaS e SaaS}}}
	\end{figure}
\end{frame}

\subsection{Tipos de nuvem}

\begin{frame}
	\frametitle{Tipos de nuvem}
	\begin{itemize}
		\item \textbf{Infraestrutura como Serviço (Iaas)}
		\item \textbf{Software como Serviço (SaaS)}
		\item \textbf{Plataforma como Serviço (PaaS)}
	\end{itemize}
\end{frame}

\begin{frame}
	\frametitle{IaaS - Infrastructure as a Service}
	\begin{itemize}
		\item O recurso fornecido ao consumidor é provisionar:
			\begin{itemize}
				\item Processamento
				\item Armazenamento
				\item Comunicação de rede
				\item Outros recursos de computação funcamentais nos quais o consumidor pode instalar e executar softwares em geral, incluindo sistemas operaionais e aplicativos
				\item Possivelmente um controle limitado de alguns componentes de rede (\textit{firewall})
			\end{itemize}
	\end{itemize}
\end{frame}

\begin{frame}
	\frametitle{PaaS - Plataform as a Service}
	\begin{itemize}
		\item O recurso fornecido ao consumidor é instalar na infraestrutura na nuvem aplicativos criados ou adiquiridos pelo consumidor
		\item O consumidor tem controle sobre as aplicações instaladas e possívelmente configuraçoes de hospedagem de aplicações
		\item O consumidor não gerencia nem controla a infraestrutura na nuvem subjacente (Rede, servidores, sistemas operacionais, armazenamento ou mesmo recursos individuais da aplicação, com a possível exeção de configurações limitadas por usuário)
	\end{itemize}
\end{frame}

\begin{frame}
	\frametitle{SaaS - Software as a Service}
	\begin{itemize}
		\item O recurso fornecido ao consumidor é o uso de aplicções do fornecedor executando em uma infraestrutura na nuvem
		\item As aplicações podem ser acessadas por vários dispositivos clientes através de interfaces leves ou ricas
		\item O consumidor não gerencia nem controla a infraestrutura na nuvem subjacente (Rede, servidores, sistemas operacionais, armazenamento ou mesmo recursos individuais da aplicação, com a possível exeção de configurações limitadas por usuário)
	\end{itemize}
\end{frame}

\begin{frame}
	\frametitle{Exemplos}
	\begin{tiny}
	\begin{table}[ht]
		\centering
		\begin{tabular}{|p{1cm}|p{5cm}|p{4cm}|}
		\hline
			Tipo & Serviço & Exemplos \\
		\hline
		\hline
		  \multirow{3}{*}{IaaS}
			& Rede virtualizada & AWS VPC, Azura Virtual Network \\
			& Armazenamento de dados & AWS S3, Google cloud storage \\
			& Servidores Virtuais & AWS EC2, Azure Virtual Machines \\
			\hline
		  \multirow{2}{*}{PaaS}
			& Infraestrutura para desenvolvimento, implantação e execução de aplicativos & Heroku, Google App Engine \\
			& Plataforma testes e gerenciamento de aplicações & AWS Elastic Beanstalk \\
			\hline
		  \multirow{3}{*}{SaaS}
			& Armazenamento Dados & DropBox, Google Drive \\
			& Editor de textos e planilha & Gsuite e Office 365 \\
			& SIstema de Gestão de banco de dados & AWS RDS, Google Cloud SQL \\
		\hline
		\end{tabular}
		\caption{Exemplos de serviços}
	\end{table}
	\end{tiny}
\end{frame}

\subsection{Modelos de implementação de nuvem}

\begin{frame}
	\frametitle{Modelos de implementação de nuvem}
	\begin{itemize}
		\item Nuvem pública
		\item Nuvem privada
		\item Nuvem híbrida
		\item Nuvem comunitária
	\end{itemize}
\end{frame}

\begin{frame}
	\frametitle{Nuvem pública}
	\begin{itemize}
		\item Provisionada para uso aberto ao público em geral
		\item Sua propriedade, gerenciamento e operação podem ser de:
			\begin{itemize}
				\item Uma empresa
				\item Uma instituição acadêmica
				\item Uma organização do governo
				\item Ou uma combinação mista
			\end{itemize}
		\item Fica nas instalações do fornecedor
	\item \textbf{OBS}: Criar uma estrutura na Amazon e configurar usando VPN ou conexão direta continua sendo uma nuvem pública
	\end{itemize}
\end{frame}

\begin{frame}
	\frametitle{Nuvem privada}
	\begin{itemize}
		\item Provisionada para uso exclusivo por uma únic organização composta de diversos consumidores
		\item A sua propriedade, gerenciamento de operação podem ser de:
			\begin{itemize}
				\item A própria organização
				\item Terceiros
				\item Combinação mista
			\end{itemize}
		\item Pode estar dentro ou fora das instalações da organização
		\item Nuvem privada não é organização e não precisa estar instalada localmente
	\end{itemize}
\end{frame}

\begin{frame}
	\frametitle{Nuvem comunitária}
	\begin{itemize}
		\item Provisionada para uso exclusivo por uma determinada comunidade de consumidores de organizações que têm interesses em comum (missão, requisitos de segurança, políticas, observância de regulamentações)
		\item A sua propriedade, gerenciamento e operação podem ser de:
			\begin{itemize}
				\item Uma organização
				\item Mais de uma organizações da comunidade
				\item Terceiros
				\item Combinação mista
			\end{itemize}
		\item Pode estar dentro ou fora das instalações das organizações participantes
	\end{itemize}
\end{frame}

\begin{frame}
	\frametitle{Nuvem híbrida}
	\begin{itemize}
		\item Composição de duas ou mais infraestruturas na nuvem (\textbf{privadas}, \textbf{comunitárias} ou \textbf{públicas}) que permanecem entidades distintas
		\item São interligadas por tecnoogia padronizada ou proprietária que permite a comunicação de dados e portabilidade de aplicações (Transferencia de processamento para a nuvem para balanceamento de carga entre nuvens)
	\end{itemize}
\end{frame}

