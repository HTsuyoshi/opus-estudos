\section{Introdução}

\begin{frame}
	\frametitle{Recursos}
	\begin{itemize}
		\item Cada recurso vai ter um \textbf{Amazon Resource name} (Identificador único)
	\end{itemize}
\end{frame}

\begin{frame}
	\frametitle{Free Tier}
	\begin{itemize}
		\item São recursos que podem ser usadas de graça na Amazon
	\end{itemize}
\end{frame}

\begin{frame}
	\frametitle{Calculadora}
	\begin{itemize}
		\item É utilizada para calcular o custo total de algum recurso
		\begin{itemize}
		\item \href{https://calculator.s3.amazonaws.com/index.html}{Calculadora antiga}
		\item \href{https://calculator.aws/}{Calculadora nova}
		\end{itemize}
	\end{itemize}
\end{frame}

\begin{frame}
	\frametitle{Regiões}
	\begin{itemize}
		\item Cada região tem um preço diferente
		\item Uma região é composta de zonas de disponibilidade
		\item Algumas regiões podem ter mais serviços que outras
		\item \textbf{OBS}: É bom saber se juridicamente a gente pode armazenar os dados fora do Brasil
			\begin{itemize}
				\item \href{https://aws.amazon.com/pt/about-aws/global-infrastructure/regions_az/}{Regiões e zonas de disponibilidade}
				\item \href{https://aws.amazon.com/pt/about-aws/global-infrastructure/regional-product-services/}{Serviços regionais}
			\end{itemize}
		\item \textbf{OBS}: Tráfegos entre zonas de disponibilidade ou regiões podem acabar sendo cobrados
	\end{itemize}
\end{frame}

\begin{frame}
	\frametitle{Zonas de disponibilidade}
	\begin{itemize}
		\item Compõem as regiões
			\begin{itemize}
				\item \href{https://aws.amazon.com/pt/about-aws/global-infrastructure/regional-product-services/}{Serviços regionais}
			\end{itemize}
		\item \textbf{OBS}: Tráfegos entre zonas de disponibilidade ou regiões podem acabar sendo cobrados
	\end{itemize}
\end{frame}

\begin{frame}
	\frametitle{Status AWS}
	\begin{itemize}
		\item Para verificar o status das zonas de disponibilidade/regiões ou recursos
			\begin{itemize}
				\item \href{http://status.aws.amazon.com/}{Status AWS}
			\end{itemize}
		\item \textbf{OBS}: Tráfegos entre zonas de disponibilidade ou regiões podem acabar sendo cobrados
	\end{itemize}
\end{frame}

\begin{frame}
	\frametitle{Pontos de presença}
	\begin{itemize}
		\item Pontos de cache utilizado pela AWS (É possível usar \textit{CNDs})
	\end{itemize}
\end{frame}

