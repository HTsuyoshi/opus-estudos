\section{AWS}

\subsection{Origem}

\begin{frame}[allowframebreaks]
	\frametitle{Origem AWS}
	\begin{itemize}
		\item 2006 - Amazon Web Services começou a oferecer infraestrutura de TI como forma de serviços web
			\begin{itemize}
				\item Low Cost (Pay-as-you-go)
				\item Agility and Instant Elasticity
				\item Open and Flexible
				\item Secure (PCI DSS Level 1, ISO27001, etc\dots)
			\end{itemize}
		\item Instância (2006):
			\begin{itemize}
				\item 1.7 GHz Xeon Processor
				\item 1.75 GB of RAM
				\item 160 GB of local disk
				\item 250 Mbps network bandwidth
			\end{itemize}
		\item Instância (2019):
			\begin{itemize}
				\item 4.0 GHz Xeon Processor (z1d instance)
				\item 24 TiB of RAM (High Memory instances)
				\item 60 TB of NVMe local storage (I3en.metal instances)
				\item 100 Gbps network bandwidth
			\end{itemize}
	\end{itemize}
\end{frame}

\subsection{Formas de acesso}

\begin{frame}
	\frametitle{Formas de acesso}
	\begin{itemize}
		\item \textbf{Console}: Permite gerenciar a infraestrutura e os recursos da aws com uma interface web
		\item \textbf{SDK}: Simplifica o uso dos serviços da AWS provendo bibliotecas para os desenvolvedores
			\begin{itemize}
				\item Tem suporte para: Java, .NET, C++, PHP, etc\dots
			\end{itemize}
		\item \textbf{CLI}: É uma ferramenta para gerenciar os serviços da AWS. É possível controlar múltiplos serviços usando a linha de comando e é possível automatizar usando scripts
	\end{itemize}
\end{frame}
