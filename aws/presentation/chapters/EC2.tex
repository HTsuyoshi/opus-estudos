\section{EC2}

\begin{frame}
	\frametitle{EC2}
	\begin{itemize}
		\item Oferece instâncias
		\item Podemos escolher: Processador, SO, Armazenamento, Redes, etc\dots
		\begin{figure}[htpb]
		\begin{center}
		\begin{tikzpicture}[scale=1, transform shape]
			\node at (0,0) {\huge M5d.xlarge};
			\node (M1) at (-2.3,-0.3) {};
			\node (M2) at (-1.6,-0.3) {};

			\draw (M1)--(M2);
			\draw ($(M1)!0.5!(M2)$)--($(M1)!0.5!(M2) - (0,0.2)$);
			\node at ($(M1)!0.5!(M2) - (0,0.4)$) {\tiny Family};

			\node (C1) at (-1.8,0.4) {};
			\node (C2) at (-1.2,0.4) {};

			\draw (C1)--(C2);
			\draw ($(C1)!0.5!(C2)$)--($(C1)!0.5!(C2) + (0,0.2)$);
			\node at ($(C1)!0.5!(C2) + (0,0.4)$) {\tiny Generation};

			\node (D1) at (-1.4,-0.3) {};
			\node (D2) at (-0.7,-0.3) {};

			\draw (D1)--(D2);
			\draw ($(D1)!0.5!(D2)$)--($(D1)!0.5!(D2) - (0,0.5)$);
			\node at ($(D1)!0.5!(D2) - (0,0.7)$) {\tiny Capabilities};

			\node (T1) at (-0.6,-0.3) {};
			\node (T2) at (2.3,-0.3) {};

			\draw (T1)--(T2);
			\draw ($(T1)!0.5!(T2)$)--($(T1)!0.5!(T2) - (0,0.2)$);
			\node at ($(T1)!0.5!(T2) - (0,0.4)$) {\tiny Size};
		\end{tikzpicture}
		\end{center}
		\caption{Tipo das instâncias}
		\label{fig:}
		\end{figure}
 		\item São de uso geral: Web/App servers, Gaming servers, Dev/Test Environments, etc\dots
	\end{itemize}
\end{frame}

\subsection{Tipo de instâncias}

\begin{frame}
	\frametitle{Instâncias de propósito geral}
	\begin{itemize}
		\item \textbf{Instâncias M5}: Equilíbrio entre memória, poder computacional e velocidade de rede
			\begin{itemize}
				\item Proporção de memória para vCPU é de 4:1
			\end{itemize}
		\item \textbf{Instâncias T3}: Tem uma linha base de performace da CPU e tem a possibilidade de passar a linha base (acumulando crédito ou pagando)
			\begin{itemize}
				\item Usado para workloads que não usam a CPU constantemente.
			\end{itemize}
		\item \textbf{Instâncias A1}: Workloads que precisam escalar em múltiplos cores, rodar instruções ARM, etc\dots
	\end{itemize}
\end{frame}

\begin{frame}
	\frametitle{Instâncias Memory-intensive workloads}
	\begin{itemize}
		\item Banco de dados de alta performace, Análise de Big Data, Cache de memória, etc\dots
		\item \textbf{R5 Instances}: Workloads que processam data sets grandes em memória
			\begin{itemize}
				\item Proporção de memória para vCPU é de 8:1
			\end{itemize}
		\item \textbf{X1/X1e Instances}: Proporção de memória para vCPU é de 16:1 e 32:1
		\item \textbf{High memory instances}: Certificado para rodar SAP HANA
			\begin{itemize}
				\item Possui 6 até 24 TB de memória
			\end{itemize}
	\end{itemize}
\end{frame}

\begin{frame}
	\frametitle{Instâncias Compute-intensive workloads}
	\begin{itemize}
		\item High-perf computing (HPC), Multiplayer Gaming, Video encoding, etc\dots
		\item \textbf{C5 Instances}: Alta performace por um preço baixo
			\begin{itemize}
				\item Proporção de memória para vCPU é 2:1
			\end{itemize}
		\item \textbf{z1d Instances}: ALta performace em uma única thread. Processador mais rápido em nuvem de 4.0 GHz
			\begin{itemize}
				\item Proporção de memória para vCPU é de 8:1
			\end{itemize}
	\end{itemize}
\end{frame}

\begin{frame}
	\frametitle{Instâncias Storage-intensive workloads}
	\begin{itemize}
		\item Uso em:
			\begin{itemize}
				\item Alta operações de I/O. Ex: High-perf databases, No SQL databases, etc\dots
				\item Muito armazenamento. Ex: Big Data, Kafka, Log processing\dots
			\end{itemize}
		\item \textbf{Instâncias I3/I3en}: Otimizadas para operações de I/O com pouca latência
		\item \textbf{Instâncias D2}: Custo baixo por armazenamento e suporta alta taxas de transferências
		\item \textbf{Instâncias H1}: Aplicações de custo baixo que usam altas transferências de dados e acesso sequencial para grandes Data Sets.
			\begin{itemize}
				\item Mais vCPUS e memória por TB que o D2
			\end{itemize}
	\end{itemize}
\end{frame}

\begin{frame}
	\frametitle{Workloads de computação acelerada}
	\begin{itemize}
		\item Machine learning, HPC, Gráficos, etc\dots
		\hfill
			\begin{figure}[htpb]
				\centering
				\includegraphics[width=0.8\textwidth]{ec2-comp-acc}
				\caption{CPU vs GPU vs FPGA vs ASCIs\cite{CDOLR}}
			\end{figure}
	\end{itemize}
\end{frame}

\begin{frame}
	\frametitle{Instâncias de computação acelerada}
	\begin{itemize}
		\item \textbf{Instâncias P2/P3}: GPU (deep learning training, HPC, etc\dots)
		\item \textbf{Instâncias G3/G4}: GPU (renderização 3D, codificação de vídeo, etc\dots)
		\item \textbf{Instâncias F1}: FPGAs programáveis (processamento de imagem, computação financeira, etc\dots)
		\item \textbf{Instâncias Inf1}: Alta performace e custo baixo para machine learning
			\begin{itemize}
				\item Integração com ML frameworks (TensorFlow, PyTorch, etc\dots)
			\end{itemize}
	\end{itemize}
\end{frame}

\begin{frame}
	\frametitle{Instâncias Bare Metal}
	\begin{itemize}
		\item Feito para workloads que não são virtualizados ou precisam de tipos específicos de hypervisors ou tem licenças que restrigem o uso de virtualização
	\end{itemize}
\end{frame}

\subsection{Amazon Machine Image (AMI)}

\begin{frame}
	\frametitle{Amazon Machine Images (AMIs)}
	\begin{itemize}
		\item Amazon Maintained
			\begin{itemize}
				\item Imagens de Windows e Linux
				\item Recebem Updates pela amazon em cada região
				\item Amazon Linux 2 (5 anos de suporte)
			\end{itemize}
		\item Marketplace Maintained
			\begin{itemize}
				\item São gerenciados e mantidos pelos parceiros da AWS
			\end{itemize}
		\item Your Machine Images
			\begin{itemize}
				\item AMIs que foram criadas de instâncias EC2
				\item Podem ser privadas, compartilhadas com outras contas ou publicadas na comunidade
			\end{itemize}
	\end{itemize}
\end{frame}

\subsection{EBS}

\begin{frame}
	\frametitle{Amazon EBS}
	\begin{itemize}
		\item Blocos de armazenamento como serviço
		\item Escolher o armazenamento e computar baseado no seu workload
		\item Pode colocar ou retirar de uma instância
		\item Volumes magnéticos ou baseados em SSD
		\item Suportam snapshots de um bloco modificado
		\item Dados criptografados por padrão em volumes EBS
		\item Fast Snapshot Restore (FSR)
		\item Rede mais otimizada para EBS em instâncias C5/C5d, M5/M5d, R5/R5d
	\end{itemize}
\end{frame}

\subsection{Security Group}

\begin{frame}
	\frametitle{Security Group}
	\begin{itemize}
		\item Firwall virtual para controlar a entrada e saída de tráfego em instâncias EC2
	\end{itemize}
\end{frame}

\subsection{Autoscaling groups}

\begin{frame}
	\frametitle{Autoscaling groups}
	\begin{itemize}
		\item Escalar horizontalmente instâncias EC2
		\item Garante que o seu grupo vai ter a quantidade desejada de instâncias
		\item Pode aumentar a capacidade em um dia e horário específico
		\item Dynamic scaling define como escalar os recursos dependendo da mudança da demanda
		\item Pode usar o CloudWatch para aumentar o número de servers usando algum parâmetro definido
		\item Health checks
	\end{itemize}
\end{frame}

\subsection{Elastic IP}

\begin{frame}
	\frametitle{Elastic IP}
	\begin{itemize}
		\item Elastic IP é um endereço IPv4 público
		\item Elastic IP é alocado para a conta da AWS e será seu até que você o libere
		\item Mascarar a falha de uma instância ou software (remapeia rapidamente o endereço para outra instância na conta)
		\item É possível especificar o endereço IP elástico em um registro DNS para o seu domínio
		\item Cobrança:
			\begin{itemize}
				\item Por hora quando um Elastic IP não está associado a uma instância em execução/encerrada ou a uma interface de rede não anexada
				\item Será cobrado por qualquer endereço IP elástico adicional associado a uma instância
			\end{itemize}
	\end{itemize}
\end{frame}

\subsection{Load Balancers}

\begin{frame}
	\frametitle{Classic Load Balancer}
	\begin{itemize}
		\item Suporte para EC2-Classic
		\item Suporte para TCP e SSL
		\item Suporte para sticky sessions usando cookies gerados pela aplicação
		\item Redireciona as requisições para instâncias registradas
		\item Tem health-checks
	\end{itemize}
\end{frame}

\begin{frame}
	\frametitle{Application Load Balancer}
	\begin{itemize}
		\item Funciona na camada de aplicação do modelo OSI (HTTP, HTTPS, gRPC)
		\item É possível adicionar regras para poder redirectionar as requisições de forma mais precisa
		\item Health checks podem ser feitos em grupos de instâncias
		\item Benefícios em relação ao clb:
			\begin{itemize}
				\item Path conditions (URL)
				\item Host conditions (Host field in http header)
				\item HTTP header conditions
				\item Multiplas aplicações em um EC2 (Bom para mircoserviços)
			\end{itemize}
	\end{itemize}
\end{frame}

\begin{frame}
	\frametitle{Netowrk Load Balancer}
	\begin{itemize}
		\item Funciona na camada de rede (TCP, UDP, TLS)
		\item Foward TCP traffic
		\item High performance
		\item Support static / Elastic IP
		\item Latency 100 ms (400 ms ALB)
	\end{itemize}
\end{frame}

\begin{frame}
	\frametitle{Gateway Load Balancer}
	\begin{itemize}
		\item Gateway + Load Balancer
			\begin{itemize}
				\item NO packet rewrite
			\end{itemize}
		\item Layer 3 load balancer
			\begin{itemize}
				\item Escalabilidade horizontal para appliances
				\item Tolerancia de falhas para appliances
				\item Transparencia de inserção de serviços
				\item Possível comparilhar em diferentes VPCs e contas
				\item Appliance as Service
			\end{itemize}
		\item Appliance de terceiros
		\item Simplifica o deployment de appliance
		\item Conectar VPCs diferentes:
			\begin{itemize}
				\item Fazer appliance ou segurança (Firewall)
			\end{itemize}
	\end{itemize}
\end{frame}

\begin{frame}
	\frametitle{Comparação}
	\begin{tiny}
	\begin{table}[htpb]
		\centering
		\caption{Fonte\footnote{\href{https://aws.amazon.com/pt/elasticloadbalancing/features/\#Product_comparisons}{Comparação dos load balancers}}}
	
		\begin{tabular}{|p{2cm}|p{2cm}|p{2cm}|p{2cm}|p{2cm}|}
			\hline
			Recurso & Application Load Balancer & Network Load Balancer & Gateway Load Balancer & Classic Load Balancer \\
			\hline \hline
			Tipo de balanceador de carga & Camada 7 & Camada 4 & Gateway da camada 3 + Balanceamento de carga da camada 4 & Camadas 4/7 \\
			\hline
			Tipo de destino & IP, instância, Lambda & IP, instância, balanceador de carga da aplicação & IP, instância & \\
			\hline
			Encerra fluxo / comportamento de proxy & Sim & Sim & Não & Sim \\
			\hline
			Listeners do protocolo & HTTP, HTTPS, gRPC & TCP, UDP, TLS & IP & TCP, SSL/TLS, HTTP, HTTPS \\
			\hline
			Acessível via & VIP & VIP & Entrada da tabela de rotas & \\
			\hline
		\end{tabular}
	\end{table}
	\end{tiny}
\end{frame}
