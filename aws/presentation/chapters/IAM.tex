\section{IAM}

\begin{frame}
	\frametitle{IAM}
	\begin{itemize}
		\item \href{https://docs.aws.amazon.com/wellarchitected/latest/security-pillar/identity-and-access-management.html}{Identity and Acess Management}
		\item Boas práticas:
			\begin{itemize}
				\item Habilitar MFA
				\item Criar um usuário padrão para cada pessoa do time e dar permissões (Não usar o \textbf{root})
				\item Usar grupos para atribuir permissões
				\item Aplicar uma política de senhas do IAM
			\end{itemize}
		\item OBS: É universal, funciona em todas as regiões
	\end{itemize}
\end{frame}

\subsection{Usuários}

\begin{frame}
	\frametitle{IAM - Users}
	\begin{itemize}
		\item \textbf{Programmatic access}
		\item Enables an access key ID and secret access key for the AWS API, CLI, SDK, and other development tools.
		\begin{itemize}
			\item Instalar o CLI para ter acesso ao AWS
			\item Dar acesso de um bucket para uma aplicação
		\end{itemize}
		\item \textit{AWS Management Console access}
		\item Enables a password that allows users to sign-in to the AWS Management Console.
		\begin{itemize}
			\item Precisa dar permissão para esse usuário
		\end{itemize}
	\end{itemize}
\end{frame}

\subsection{Tags}

\begin{frame}
	\frametitle{IAM - Tags}
	\begin{itemize}
		\item Servem para a gente identificar serviços
		\item É possível fazer um relatório de faturamento baseado em \textit{Tags}
		\item \textbf{OBS}: É possível ter até 50 tags por serviço
	\end{itemize}
\end{frame}

\subsection{Policies}

\begin{frame}
	\frametitle{IAM - Políticas Pt.1}
	\begin{itemize}
		\item É uma boa prática criar grupos com permissões para os usuários. E não colocar permissões diretamente no usuário
		\item Permissões mais específicas são mais fortes (Permissão de usuário prevalece contra permissão de grupo)
		\item \href{https://docs.aws.amazon.com/pt_br/IAM/latest/UserGuide/id_credentials_passwords_account-policy.html}{Políticas de senha}
			\begin{itemize}
				\item Exigir que o usuário use senhas fortes
				\item Expiração de senhas
				\item Impedir reutilização de senhas
				\item Etc\dots
			\end{itemize}
	\end{itemize}
\end{frame}

\begin{frame}
	\frametitle{IAM - Políticas Pt.2}
	\begin{itemize}
		\item Políticas de acesso
			\begin{itemize}
				\item As políticas podem ser definidas por um arquivo Json
				\item Pode ser usado políticas prontas ou criar políticas específicas
				\item Então as políticas podem ser atribuídas em usuários/grupos
			\end{itemize}
	\end{itemize}
\end{frame}

\subsection{Roles}

\begin{frame}
	\frametitle{Funções/Roles}
	\begin{itemize}
		\item Dar permissões para:
			\begin{itemize}
				\item Recursos
				\begin{itemize}
					\item Ex: Dar permissão para uma instância acessar um bucket
				\end{itemize}
				\item Outras contas AWS
				\item Federações do SAML 2.0
				\item Identidade web (Login Google, amazon, etc\dots)
			\end{itemize}
	\end{itemize}
\end{frame}

\subsection{Relatórios}

\begin{frame}
	\frametitle{Relatórios de acesso}
	\begin{itemize}
		\item Relatórios de credenciais
			\begin{itemize}
				\item Lista de todas as credenciais geradas
			\end{itemize}
		\item Access Analyzer: Gera um relatório de políticas pra a gente ver o que precisa ser modificado. é possível arquivar, resolver, etc\dots
	\end{itemize}
\end{frame}
