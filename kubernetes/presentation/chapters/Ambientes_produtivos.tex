\section{Ambientes Produtivos}

\begin{frame}
\frametitle{Disponibilidade e Escalabilidade}

Para criar um \textit{cluster} de alta disponibilidade:

\begin{itemize}
	\item Separar \textit{control plane} do nó dos \textit{workers}
	\item Replicar os componentes do \textit{control plane} em multiplos nós
	\item Balancear o tráfico de rede na \textit{API} do \textit{cluster}
	\item Ter nós \textit{workers} suficientes ou ter nós que conseguem ficar disponíveis rapidamente
\end{itemize}

Escalabilidade

\begin{itemize}
	\item Ser capaz de receber as demandas que a aplicação recebe normalmente
	\item Ser capaz de aguentar as demandas em eventos especiais ou datas comemorativas
	\item Planejar como escalar ou reduzir os recursos do \textit{control plane} e dos \textit{workers}
\end{itemize}
\end{frame}

\begin{frame}
\frametitle{\textit{Control plane} em produção}
\begin{itemize}
	\item Gerenciar certificados
	\item \uncover<2->{\textbf{Alta disponibilidade}: Estender o \textit{control plane}}
		\begin{itemize}
			\item \uncover<3->{\textit{Load Balancer} para o \textbf{apiserver}}
			\item \uncover<4->{\textit{Backup} do \textbf{etcd}}
			\item \uncover<5->{Múltiplos control planes em máquinas separadas}
			\item \uncover<6->{Espalhar por múltiplos \textit{data centers}. (Continua disponível mesmo que uma zona fique fora do ar)}
			\item \uncover<7->{Gerenciar \textit{cluster}: Certificados, fazer \textit{upgrade}, etc\dots}
		\end{itemize}
\end{itemize}
\end{frame}

\begin{frame}
\frametitle{\textit{Worker Nodes} em produção}
\begin{itemize}
	\item Gerenciar os recursos corretamente (Colocar a quantidade apropriada de Memória, CPU e armazenamento nos nós)
	\item \uncover<2->{Validar se os nós tem os recursos necessários para entrar no cluster}
	\item \uncover<3->{Escalabilidade dos nós}
	\item \uncover<4->{\textit{Health checks} para verificar se os nós estão saudáveis}
		\begin{itemize}
			\item \uncover<5->{\textbf{Node Problem Detector}: Pode ser rodado como um \textit{daemon} ou como um \textit{daemonset}}
			\item \uncover<6->{Usam vários \textit{daemons} para coletar informações e reportam essas condições usando as condições do nó ou eventos.}
		\end{itemize}
\end{itemize}
\end{frame}

\begin{frame}
\frametitle{\textit{Usuários} no ambiente de produção}
\begin{itemize}
	\item \textbf{Autenticação}: Autenticar os usuários do \textbf{apiserver} com certificados, tokens, proxy de autenticação ou HTTP
		\begin{itemize}
			\item \uncover<2->{É possível usar \textbf{Kerberos} ou \textbf{LDAP} para uma autenticação mais avançada}
		\end{itemize}
	\item \uncover<3->{\textbf{Autorização}: \textbf{RBAC} ou \textbf{ABAC}}
		\begin{itemize}
			\item \uncover<4->{\textbf{RBAC (\textit{Role-based access control})}: Permissões para namespaces específicos (\textit{Role} ou para o \textit{cluster} inteiro). Com \textbf{RoleBindings} e \textbf{ClusterBindings} é possível relacionar com usuários específicos}
				\begin{itemize}
					\item \uncover<5->{Criar certificados para usuários \textbf{CertificateSigningRequest}}
				\end{itemize}
			\item \uncover<6->{\textbf{ABAC (\textit{Attribute-based access control})}: Criar políticas baseadas em atributos do \textit{cluster}.}
				\begin{itemize}
					\item \uncover<7->{Usuários podem acessar determinados tipos de atributos (\textit{pod}, \textit{namespaces}, \textit{apiGroups})}
				\end{itemize}
		\end{itemize}
\end{itemize}
\end{frame}
