\section{Recursos}

\begin{frame}
\frametitle{Recursos - Namespaces}

\begin{itemize}
	\item Os \textit{namespaces} permitem isolar grupos de recursos em um cluster
		\begin{itemize}
			\item \uncover<2->{Os \textit{namespaces} são uma forma de dividir o cluster para várias pessoas com limite de recursos.\footnote{\href{https://kubernetes.io/docs/concepts/overview/working-with-objects/namespaces/}{Kubernetes Documentation}}}
			\item \uncover<3->{Os administradores do cluster devem dividir os recursos}
		\end{itemize}
	\item \uncover<4->{O nome dos recursos precisam ser únicos em cada namespace}
	\item \uncover<5->{Não são todos os objetos que aceitam o isolamento do \textit{namespace}}
\end{itemize}
\end{frame}

\begin{frame}[containsverbatim]
\frametitle{Recursos - Namespaces}

Para ver os recursos que são e não são isolados por namespace

\begin{center}
\begin{minipage}{0.9\textwidth}
\begin{block}{\textbf{values.yaml}}
\begin{lstlisting}[language=bash]
# In a namespace
kubectl api-resources --namespaced=true

# Not in a namespace
kubectl api-resources --namespaced=false
\end{lstlisting}
\end{block}
\end{minipage}
\end{center}
\end{frame}

\begin{frame}
\frametitle{Namespaces}
\centering
\begin{multicols}{3}
	\begin{tikzpicture}
		\makens{site}{ctfd}
		\node (L1) at ($(site) + (0,1.1)$) {\tiny site-hpa};
		\node (L2) at ($(L1) - (0,0.3)$) {\tiny site-deployment};
		\node (L3) at ($(L2) - (0,0.3)$) {\tiny site-service};
		\node (L4) at ($(L3) - (0,0.3)$) {\tiny site-ingress};
		\node (L1) at ($(ctfd) + (0,1.1)$) {\tiny secrets (mariaDB)};
		\node (L2) at ($(L1)  - (0,0.3)$) {\tiny ctfd-202{1,2}-service};
		\node (L3) at ($(L2)  - (0,0.3)$) {\tiny ctfd-202{1,2}-deployment};
	\end{tikzpicture}
	\columnbreak
	\begin{tikzpicture}
		\makens{database}{redis}
		\node (L1) at ($(database) + (0,1.1)$) {\tiny secrets (mariaDB)};
		\node (L2) at ($(L1) - (0,0.3)$) {\tiny db-202\{1,2\}-deployment};
		\node (L3) at ($(L2) - (0,0.3)$) {\tiny db-202\{1,2\}-service};
		\node (L1) at ($(redis) + (0,1.1)$) {\tiny redis-202\{1,2\}-statefulset};
		\node (L2) at ($(L1) - (0,0.3)$) {\tiny redis-202\{1,2\}-service};
	\end{tikzpicture}
	\columnbreak
	\begin{tikzpicture}
	\makens{challs}{ingress-nginx}
		\node (L1) at ($(challs) + (0,1.1)$) {\tiny \{uccdi,xor-otp\}-hpa};
		\node (L2) at ($(L1) - (0,0.3)$) {\tiny \{uccdi,xor-otp\}-deployment};
		\node (L3) at ($(L2) - (0,0.3)$) {\tiny \{uccdi,xor-otp\}-service};
		\node (L4) at ($(L3) - (0,0.3)$) {\tiny uccdi-ingress};
		\node (L1) at ($(ingress-nginx) + (0,1.1)$) {\tiny ingress-nginx-controller};
	\end{tikzpicture}
\end{multicols}
\end{frame}

\begin{frame}
\frametitle{Recursos - Daemonsets}
\begin{itemize}
	\item O \textit{DaemonSet} garante que todos os nós ou os nós selecionados rodem uma cópia de um \textit{Pod}
	\item \uncover<2->{Quando nós são adicionados nos clusters, os \textit{Pods} são adicionados nesses nós e vice versa}
	\item \uncover<3->{Deletar o \textit{DaemonSet} mata os \textit{Pods} que foram criados}
	\item \uncover<4->{Alguns exemplos de uso:}
	\begin{itemize}
		\item \uncover<4->{Rodar um \textit{cluster storage daemon} em todos os nós}
		\item \uncover<5->{Rodar um \textit{logs collection daemon} em todos os nós}
		\item \uncover<6->{Rodar um \textit{node monitoring daemon} em todos os nós}
	\end{itemize}
\end{itemize}
\end{frame}

\begin{frame}
\frametitle{Recursos - ConfigMaps}
\begin{itemize}
	\item O \textit{ConfigMap} é usado para guardar dados não confidenciais em pares de chave-valor
	\item \uncover<2->{Pode ser usado como:}
		\begin{itemize}
			\item \uncover<3->{Variáveis de ambiente}
			\item \uncover<4->{Argumentos de comando de linha}
			\item \uncover<5->{Arquivos de configuração em um volume}
		\end{itemize}
\end{itemize}
\end{frame}

\begin{frame}[containsverbatim]
\frametitle{Recursos - ConfigMaps}
\begin{itemize}
	\item Aplicação no projeto prático:
\begin{center}
\begin{minipage}{0.9\textwidth}
\begin{block}{Variáveis de ambiente}
\begin{lstlisting}
apiVersion: v1
kind: ConfigMap
metadata:
  name: ctfd-env-configmap
  namespace: ctfd
data:
  DB_FILE: '/etc/ctfd-db/database'
  DB_USER_FILE: '/etc/ctfd-db/username'
  DB_PASS_FILE: '/etc/ctfd-db/password'
  ACCESS_LOG: '-'
  ERROR_LOG: '-'
  UPLOAD_FOLDER: '/var/uploads'
  LOG_FOLDER: '/var/log/CTFd'
  REVERSE_PROXY: 'true'
\end{lstlisting}
\end{block}
\end{minipage}
\end{center}
\end{itemize}
\end{frame}

\begin{frame}
\frametitle{Recursos - Secrets}

\begin{itemize}
	\item O \textit{secret} é um objeto que contém uma pequena quantidade de dados \textbf{sensíveis}
	\item \uncover<2->{Como senhas, tokens, chaves, etc\dots}
	\item \uncover<3->{Dessa forma não é preciso incluir dados confidenciais no seu código}
	\item \uncover<4->{Algumas boas práticas para manter o \textit{secrets} seguro:\footnote{\url{https://kubernetes.io/docs/concepts/security/secrets-good-practices/}}}
		\begin{itemize}
			\item \uncover<4->{Encryption at Rest for Secrets}
			\item \uncover<5->{Enable or configure RBAC rules with least-privilege access to Secrets}
			\item \uncover<6->{Restrict Secret access to specific containers}
			\item \uncover<7->{Consider using external Secret store providers}
		\end{itemize}
\end{itemize}
\end{frame}

\begin{frame}[containsverbatim]
\frametitle{Recursos - Secrets}
\begin{itemize}
	\item Aplicação no projeto prático:
\begin{center}
\begin{minipage}{0.9\textwidth}
\begin{block}{Volumes}
\begin{lstlisting}
apiVersion: v1
kind: Secret
metadata:
  name: ctfd-db-2022
  namespace: database
type: Opaque
data:
  database: 'Y3RmZGRiCg=='
  username: 'Y3RmZHVzZXIK'
  password: 'Y3RmZHBhc3MK'
  root-pass: 'Y3RmZHJvb3RwYXNzCg=='
\end{lstlisting}
\end{block}
\end{minipage}
\end{center}
\end{itemize}
\end{frame}

\begin{frame}
\frametitle{Recursos - Services}
\begin{itemize}
	\item Os \textbf{Services} conseguem abstrair a parte da rede dos \textbf{Pods}
	\item \uncover<2->{Os \textbf{Services} cuidam da parte do balanceamento de carga e \textbf{Service Discovery}}
\end{itemize}
\end{frame}

\begin{frame}[containsverbatim]
\frametitle{Recursos - ClusterIP}
\begin{itemize}
	\item Expõe o serviço para o \textit{cluster} interno
	\item O serviço só pode ser alcançado dentro do cluster
\begin{center}
\begin{minipage}{0.9\textwidth}
\begin{block}{\textbf{ClusterIP} do Apache}
\scriptsize
\begin{lstlisting}[language=bash]
apiVersion: v1
kind: Service
metadata:
  name: ctfd-site-service
  namespace: site
spec:
  type: ClusterIP
  ports:
  - name: 'http'
    port: 80
    targetPort: 80
  selector:
    ctfd.site.pod: ctfd-site-pod
\end{lstlisting}
\end{block}
\end{minipage}
\end{center}
\end{itemize}
\end{frame}

\begin{frame}
\frametitle{Recursos - NodePort}
\begin{itemize}
	\item Expõe o serviço em todos os Nós em uma porta estática
	\item \uncover<2->{\textbf{node port} o K8s aloca uma porta com o protocolo do serviço (TCP, UDP, SCTP, etc\dots)}
	\item \uncover<3->{Todo os nós começam a ouvir na porta especificada e redirecionar o tráfico para o \textit{endpoint} do serviço}
\end{itemize}
\end{frame}

\begin{frame}
\frametitle{Recursos - LoadBalancer}
\begin{itemize}
	\item Expõe o serviço usando o provedor de nuvem especificado
	\item \uncover<2->{O tráfico de rede externo do \textit{Load Balancer} é redirecionado para os \textbf{Pods}. A distribuição dos requests é decidida pelo provedor}
\end{itemize}
\end{frame}

\begin{frame}[containsverbatim]
\frametitle{Recursos - ExternalName}
\begin{itemize}
	\item Mapeia o serviço para um nome \textbf{DNS}
\begin{center}
\begin{minipage}{0.9\textwidth}
	\begin{block}{\textbf{ExternalName} uccdi}
\tiny
\begin{lstlisting}[language=bash]
apiVersion: v1
kind: Service
metadata:
  name: ctfd-2022-chall-uccdi
  namespace: challs
spec:
  type: ExternalName
  externalName: ctfd-2022-chall-uccdi-deployment.default.svc.cluster.local
\end{lstlisting}
\end{block}
\end{minipage}
\end{center}
\end{itemize}
\end{frame}

\begin{frame}
\frametitle{Recursos - Ingress}
\begin{itemize}
	\item \textbf{Ingress}: Gerencia acesso externo para os serviços dentro do cluster (Geralmente o protocolo \textbf{HTTP})
		\begin{itemize}
			\item \uncover<2->{Provê balanceamento de carga, hosts virtuais e pode ser um terminal SSL/TLS}
			\item \uncover<3->{O ingress não expõe portas ou protocolos}
				\begin{itemize}
					\item \uncover<4->{Serviços que não são \textbf{HTTP} ou \textbf{HTTPS} são expostos usando \textbf{Serviços} \textbf{Nodeport} ou \textbf{LoadBalancer}}
				\end{itemize}
		\end{itemize}
	\item \uncover<5->{\textbf{Ingress Controllers}: É necessário para que o as regras o \textbf{Ingress} funcionem}
	\item \uncover<6->{Alguns exemplos são os \textbf{Ingress Controllers} da AWS, GCE e nginx\footnote{\href{https://kubernetes.io/docs/concepts/services-networking/ingress-controllers/}{Outros exemplos de Ingress-Controllers}}}
\end{itemize}
\end{frame}

\begin{frame}
\frametitle{Recursos - Persistência de dados}
\begin{itemize}
	\item \textbf{Persistent Volume (PV)}: é um armazenameto do cluster que foi disponibilizado pelo administrador ou dinamicamente disponibilizado usando \textbf{Storage Class}
		\begin{itemize}
			\item \uncover<2->{É um recurso assim como um \textbf{Node}}
			\item \uncover<3->{Tem um ciclo de vida independente de um \textbf{Pod} que usa o \textbf{PV}}
			\item \uncover<4->{Esse objeto pega os detalhes de implementação do armazenamento disponibilizado. (Podendo ser NFS, iSCSI, cloud, etc\dots)}
		\end{itemize}
	\item \uncover<5->{\textbf{Persistent Volume Claim}: Permite um usuário consumir armazenamentos}
		\begin{itemize}
			\item \uncover<6->{É comum o usuário precisar de \textbf{PVs} com várias propriedades, tamanhos e acessos}
		\end{itemize}
\end{itemize}
\end{frame}

\begin{frame}
\frametitle{Recursos - Lifecycle}
Provisionamento:
\begin{itemize}
	\item \textbf{Estático}: O administrador cria um número de \textbf{PVs} e vão ficar disponíveis para consumo
	\item \uncover<2->{\textbf{Dinâmico}: Quando nenhum \textbf{PV} estático supre as necessidades do \textbf{PVC}, o cluster dinâmicamente provê um volume especialmente para o \textbf{PVC}}
		\begin{itemize}
			\item \uncover<3->{O \textbf{PVC} precisa pedir por uma \textbf{Storage Class} que foi configurada para a alocação dinâmica acontecer}
		\end{itemize}
\end{itemize}
\end{frame}

\begin{frame}
\frametitle{Recursos - Lifecycle}
\begin{itemize}
	\item Binding:
	\begin{itemize}
		\item O \textbf{PVC} só vai juntar com um \textbf{PV} se o \textbf{PV} foi provisionado dinâmicamente.
		\item \uncover<2->{Caso não seja provisionado dinâmicamente vai pegar o \textbf{PV} com pelo menos a capacidade requerida pelo \textbf{PVC}}
		\item \uncover<3->{Um \textbf{PVC} só vai se juntar com um \textbf{PV} e vice versa}
	\end{itemize}
	\item \uncover<4->{Using:}
	\begin{itemize}
		\item \uncover<5->{Então os \textbf{Pods} usam os \textbf{Claims} como volumes}
		\item \uncover<6->{O \textbf{PV} então perntece ao usuário até ele parar de usar}
	\end{itemize}
\end{itemize}
\end{frame}

\begin{frame}
\frametitle{Recursos - Lifecycle}
\begin{itemize}
	\item Storage Object in Use Protection
		\begin{itemize}
			\item \uncover<2->{Ele garante que \textbf{PVC} ativos em um \textbf{Pod} e um \textbf{PV} não podem ser removido do sistema}
			\item \uncover<3->{Só vai ser removido quando nenhum \textbf{Pod} usar o \textbf{PVC}}
			\item \uncover<4->{Um \textbf{PV} só vai ser deletado quando não estiver junto com nenhum \textbf{PVC}}
		\end{itemize}
\end{itemize}
\end{frame}

\begin{frame}
\frametitle{Recursos - Lifecycle}
\begin{itemize}
	\item Reclaiming:
		\begin{itemize}
			\item \uncover<2->{Retain}
				\begin{itemize}
					\item \uncover<3->{Depois de um \textbf{PVC} for deletado, o \textbf{PV} mantém seus dados e fica indisponível para outros \textbf{PVC}}
					\item \uncover<4->{O administrador precisa alocar ele manualmente}
				\end{itemize}
			\item \uncover<5->{Delete}
				\begin{itemize}
					\item \uncover<6->{Remove o \textbf{PV} e o armazenamento externo (AWS EBS, GCE PD, etc\dots)}
					\item \uncover<7->{}
				\end{itemize}
		\end{itemize}
\end{itemize}
\end{frame}

\begin{frame}
\frametitle{Recursos - Jobs e CronJobs}
\begin{itemize}
	\item \textbf{Jobs}: Cria um ou mais \textit{Pods} para executar uma tarefa
		\begin{itemize}
			\item \uncover<2->{Continua reexecutando os \textit{Pods} até que o número especificado de tentativas dele termine (\textit{backoff failure policy})}
			\item \uncover<3->{Assim que um \textit{Pod} completa com sucesso, o \textit{Job} continua a executar até o número de sucessos seja atingido}
			\item \uncover<4->{Suspender/Apagar um \textbf{Job} vai apagar os \textit{Pods} criados.}
			\item \uncover<5->{Quando um \textbf{Job} termina os \textit{Pods} e o \textbf{Job} em si não são apagados}
			\item \uncover<6->{É possível rodar \textbf{Jobs} em paralelo}
		\end{itemize}
	\item \uncover<7->{\textbf{Cronjobs}: Executa uma tarefa periodicamente em um determinado cronograma, escrito no formato Cron.}
		\begin{itemize}
			\item \uncover<7->{O fuso horário para o contêiner executando o \textbf{kube-controller-manager} determina o fuso horário que o \textbf{CronJob} utiliza}
			\item \uncover<8->{O nome não pode ter mais que 52 caracteres}
			\item \uncover<9->{Podem fazer backup, enviar emails, etc\dots}
		\end{itemize}
\end{itemize}
\end{frame}

\begin{frame}
\frametitle{Recursos - StatefulSet}
\begin{itemize}
	\item O \textbf{StatefulSet} é um objeto da \textit{API} que é usado para gerenciar aplicações com estados
	\item \uncover<2->{Gerencia e escala os \textbf{Pods} de forma previsível (Nomes únicos e com ordenados)}
	\item \uncover<3->{Diferente do \textbf{Deployment} o \textbf{StatefulSet} mantém um indentificador para cada \textbf{Pod}}
	\item \uncover<4->{Os \textbf{Pods} são criados da mesma especificação, mas não podem ser trocados}
	\item \uncover<5->{Os indentificadores dos \textbf{Pods} facilitam na hora de relacionar um \textbf{Pod} com seu \textbf{Volume}}
\end{itemize}
\end{frame}

\begin{frame}
\frametitle{Recursos - HPA}
	\textit{Horizontal Pod Autoscaling}
	\begin{itemize}
		\item Automaticamente atualiza recursos (como \textit{Deployment} ou \textit{StatefulSet}), tem o objetivo de escalar automaticamente de acordo com a demanda
		\item \uncover<2->{Ou seja, mais \textit{pods} \textbf{iguais} vão ser criados}
		\item \uncover<3->{Quando a carga da aplicação abaixar os \textit{pods} vão se reduzir até a quantidade mínima}
		\item \uncover<4->{Não se aplica a objetos que não podem ser escalados (Exemplo: \textbf{DaemonSet})}
		\item \uncover<5->{Periodicamente o \textbf{HPA} ajusta os \textbf{pods} do seu objeto observando métricas (CPU, memória ou métricas customizadas)}
	\end{itemize}
\end{frame}

\begin{frame}
\frametitle{Recursos - RBAC}
\begin{itemize}
	\item \textbf{Role-based access control (RBAC)} é um método de regular acesso para o computador ou recursos na rede baseado nos seus papéis
	\item \uncover<2->{O \textbf{RBAC} usa a \textit{API} \textbf{rbac.authorization.k8s.io} para autorizar as decisões}
	\item \uncover<3->{É possível configurar as políticas dinamicamentes pela API do kuberntes}
\end{itemize}
\end{frame}

\begin{frame}
\frametitle{Recursos - RBAC}
\begin{itemize}
	\item \textbf{RBAC Role}
		\begin{itemize}
			\item \uncover<2->{Permissões dentro de um \textbf{namespace}}
		\end{itemize}
	\item \uncover<3->{\textbf{RBAC Cluster Role}}
		\begin{itemize}
			\item \uncover<3->{Permissões que se aplicam em todos os \textbf{namespaces}}
		\end{itemize}
	\item \uncover<4->{As regras do \textbf{RBAC Role} e do \textbf{RBAC ClusterRole} são sempre aditivas}
\end{itemize}
\end{frame}

\begin{frame}[containsverbatim]
\frametitle{Recursos - CRD}
\begin{itemize}
	\item \textbf{Custom Resources} são extensões da \textit{API} do kubernetes
	\item \textbf{Custom Resources Definition} é uma forma simples de criar um \textbf{Custom Resources}
\begin{center}
\begin{minipage}{0.9\textwidth}
\begin{block}{Calico exemplo}
\scriptsize
\begin{lstlisting}
apiVersion: projectcalico.org/v3
kind: GlobalNetworkPolicy
metadata:
  name: default-deny
spec:
  selector: projectcalico.org/namespace != "kube-system"
  types:
  - Ingress
  - Egress
\end{lstlisting}
\end{block}
\end{minipage}
\end{center}
\end{itemize}
\end{frame}
