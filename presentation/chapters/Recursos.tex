\section{Recursos}

\begin{frame}
\frametitle{Recursos - Namespaces}

\begin{itemize}
	\item Os \textit{namespaces} permitem isolar grupos de recursos em um cluster
		\begin{itemize}
			\item \uncover<2->{Os \textit{namespaces} são uma forma de dividir o cluster para várias pessoas com limite de recursos.\footnote{\href{https://kubernetes.io/docs/concepts/overview/working-with-objects/namespaces/}{Kubernetes Documentation}}}
			\item \uncover<3->{Os administradores do cluster devem dividir os recursos}
		\end{itemize}
	\item \uncover<4->{O nome dos recursos precisam ser únicos em cada namespace}
	\item \uncover<5->{Não são todos os objetos que aceitam o isolamento do \textit{namespace}}
\end{itemize}

Recursos (conceito e prática: comandos e aplicação de yaml)
    Namespaces
    Deployments
    Daemonsets - conceito
    Configmaps
    Secrets
    Services
    Ingress
    Persistência de dados (Storageclass - SC, Persistent Volume Claim - PVC, Persistent Volume - PV)
    Job e Cronjobs - conceito
    Statefulsets - conceito
    HPA - conceito
    RBAC (roles, service accounts, users) - conceito
    CRD - conceito

\end{frame}

\begin{frame}[containsverbatim]
\frametitle{Recursos - Namespaces}

Para ver os recursos que estão e não são gerenciados por namespace

\begin{center}
\begin{minipage}{0.9\textwidth}
\begin{block}{\textbf{values.yaml}}
\begin{lstlisting}[language=bash]
# In a namespace
kubectl api-resources --namespaced=true

# Not in a namespace
kubectl api-resources --namespaced=false
\end{lstlisting}
\end{block}
\end{minipage}
\end{center}
\end{frame}

\begin{frame}
\frametitle{Namespaces}
\centering
\begin{multicols}{3}
	\begin{tikzpicture}
		\makens{site}{ctfd}
		\node (L1) at ($(site) + (0,1.1)$) {\tiny site-hpa};
		\node (L2) at ($(L1) - (0,0.3)$) {\tiny site-deployment};
		\node (L3) at ($(L2) - (0,0.3)$) {\tiny site-service};
		\node (L4) at ($(L3) - (0,0.3)$) {\tiny site-ingress};
		\node (L1) at ($(ctfd) + (0,1.1)$) {\tiny secrets (mariaDB)};
	\end{tikzpicture}
	\columnbreak
	\begin{tikzpicture}
		\makens{database}{redis}
		\node (L1) at ($(database) + (0,1.1)$) {\tiny secrets (mariaDB)};
		\node (L2) at ($(L1) - (0,0.3)$) {\tiny db-202\{1,2\}-statefulset};
		\node (L3) at ($(L2) - (0,0.3)$) {\tiny db-202\{1,2\}-service};
		\node (L1) at ($(redis) + (0,1.1)$) {\tiny secrets (mariaDB)};
		\node (L2) at ($(L1) - (0,0.3)$) {\tiny db-202\{1,2\}-statefulset};
		\node (L3) at ($(L2) - (0,0.3)$) {\tiny db-202\{1,2\}-service};
	\end{tikzpicture}
	\columnbreak
	\begin{tikzpicture}
	\makens{challs}{ingress-nginx}
		\node (L1) at ($(challs) + (0,1.1)$) {\tiny \{uccdi,xor-otp\}-hpa};
		\node (L2) at ($(L1) - (0,0.3)$) {\tiny \{uccdi,xor-otp\}-deployment};
		\node (L3) at ($(L2) - (0,0.3)$) {\tiny \{uccdi,xor-otp\}-service};
		\node (L4) at ($(L3) - (0,0.3)$) {\tiny uccdi-ingress};
		\node (L1) at ($(ingress-nginx) + (0,1.1)$) {\tiny ingress-nginx-controller};
		\node (L2) at ($(L1) - (0,0.3)$) {\tiny ingress-dns};
	\end{tikzpicture}
\end{multicols}
\end{frame}

\begin{frame}
\frametitle{Recursos - Daemonsets}
\begin{itemize}
	\item O \textit{DaemonSet} garante que todos os nós ou os nós selecionados rodem uma cópia de um \textit{Pod}
	\item \uncover<2->{Quando nós são adicionados nos clusters, os \textit{Pods} são adicionados nesses nós e vice versa}
	\item \uncover<3->{Deletar o \textit{DaemonSet} mata os \textit{Pods} que foram criados}
	\item \uncover<4->{Alguns exemplos de uso:}
	\begin{itemize}
		\item \uncover<4->{Rodar um \textit{cluster storage daemon} em todos os nós}
		\item \uncover<5->{Rodar um \textit{logs collection daemon} em todos os nós}
		\item \uncover<6->{Rodar um \textit{node monitoring daemon} em todos os nós}
	\end{itemize}
\end{itemize}
\end{frame}

\begin{frame}
\frametitle{Recursos - ConfigMaps}
\begin{itemize}
	\item O \textit{ConfigMap} é usado para guardar dados não confidenciais em pares de chave-valor
	\item \uncover<2->{Pode ser usado como:}
		\begin{itemize}
			\item \uncover<3->{Variáveis de ambiente}
			\item \uncover<4->{Argumentos de comando de linha}
			\item \uncover<5->{Arquivos de configuração em um volume}
		\end{itemize}
\end{itemize}
\end{frame}

\begin{frame}[containsverbatim]
\frametitle{Recursos - ConfigMaps}
\begin{itemize}
	\item Aplicação no projeto prático:
\begin{center}
\begin{minipage}{0.9\textwidth}
\begin{block}{Variáveis de ambiente}
\begin{lstlisting}
apiVersion: v1
kind: ConfigMap
metadata:
  name: ctfd-env-configmap
  namespace: ctfd
data:
  DB_FILE: '/etc/ctfd-db/database'
  DB_USER_FILE: '/etc/ctfd-db/username'
  DB_PASS_FILE: '/etc/ctfd-db/password'
  ACCESS_LOG: '-'
  ERROR_LOG: '-'
  UPLOAD_FOLDER: '/var/uploads'
  LOG_FOLDER: '/var/log/CTFd'
  REVERSE_PROXY: 'true'
\end{lstlisting}
\end{block}
\end{minipage}
\end{center}
\end{itemize}
\end{frame}

\begin{frame}
\frametitle{Recursos - Secrets}

\begin{itemize}
	\item O \textit{secret} é um objeto que contém uma pequena quantidade de dados \textbf{sensíveis}
	\item \uncover<2->{Como senhas, tokens, chaves, etc\dots}
	\item \uncover<3->{Dessa forma não é preciso incluir dados confidenciais no seu código}
	\item \uncover<4->{Algumas boas práticas para manter o \textit{secrets} seguro:\footnote{\url{https://kubernetes.io/docs/concepts/security/secrets-good-practices/}}}
		\begin{itemize}
			\item \uncover<4->{Encryption at Rest for Secrets}
			\item \uncover<5->{Enable or configure RBAC rules with least-privilege access to Secrets}
			\item \uncover<6->{Restrict Secret access to specific containers}
			\item \uncover<7->{Consider using external Secret store providers}
		\end{itemize}
\end{itemize}
\end{frame}

\begin{frame}[containsverbatim]
\frametitle{Recursos - Secrets}
\begin{itemize}
	\item Aplicação no projeto prático:
\begin{center}
\begin{minipage}{0.9\textwidth}
\begin{block}{Volumes}
\begin{lstlisting}
apiVersion: v1
kind: Secret
metadata:
  name: ctfd-db-2022
  namespace: database
type: Opaque
data:
  database: 'Y3RmZGRiCg=='
  username: 'Y3RmZHVzZXIK'
  password: 'Y3RmZHBhc3MK'
  root-pass: 'Y3RmZHJvb3RwYXNzCg=='
\end{lstlisting}
\end{block}
\end{minipage}
\end{center}
\end{itemize}
\end{frame}

\begin{frame}
\frametitle{Recursos - Services}
    Secrets
\end{frame}

\begin{frame}
\frametitle{Recursos - Ingress}
    Secrets
\end{frame}

\begin{frame}
\frametitle{Recursos - Persistência de dados}
    Secrets
\end{frame}

\begin{frame}
\frametitle{Recursos - Jobs e CronJobs}
\begin{itemize}
	\item \textbf{Jobs}: Cria um ou mais \textit{Pods} para executar uma tarefa
		\begin{itemize}
			\item \uncover<2->{Continua reexecutando os \textit{Pods} até que o número especificado de tentativas dele termine (\textit{backoff failure policy})}
			\item \uncover<3->{Assim que um \textit{Pod} completa com sucesso, o \textit{Job} continua a executar até o número de sucessos seja atingido}
			\item \uncover<4->{Suspender/Apagar um \textbf{Job} vai apagar os \textit{Pods} criados.}
			\item \uncover<5->{Quando um \textbf{Job} termina os \textit{Pods} e o \textbf{Job} em si não são apagados}
			\item \uncover<6->{É possível rodar \textbf{Jobs} em paralelo}
		\end{itemize}
	\item \uncover<7->{\textbf{Cronjobs}: Executa uma tarefa periodicamente em um determinado cronograma, escrito no formato Cron.}
		\begin{itemize}
			\item \uncover<7->{O fuso horário para o contêiner executando o \textbf{kube-controller-manager} determina o fuso horário que o \textbf{CronJob} utiliza}
			\item \uncover<8->{O nome não pode ter mais que 52 caracteres}
			\item \uncover<9->{Podem fazer backup, enviar emails, etc\dots}
		\end{itemize}
\end{itemize}
\end{frame}

\begin{frame}
\frametitle{Recursos - StatefulSet}
    Secrets
\end{frame}

\begin{frame}
\frametitle{Recursos - HPA}
	\textit{Horizontal Pod Autoscaling}
	In Kubernetes, a HorizontalPodAutoscaler automatically updates a workload resource (such as a Deployment or StatefulSet), with the aim of automatically scaling the workload to match demand.

Horizontal scaling means that the response to increased load is to deploy more Pods. This is different from vertical scaling, which for Kubernetes would mean assigning more resources (for example: memory or CPU) to the Pods that are already running for the workload.

If the load decreases, and the number of Pods is above the configured minimum, the HorizontalPodAutoscaler instructs the workload resource (the Deployment, StatefulSet, or other similar resource) to scale back down.

Horizontal pod autoscaling does not apply to objects that can't be scaled (for example: a DaemonSet.)

The HorizontalPodAutoscaler is implemented as a Kubernetes API resource and a controller. The resource determines the behavior of the controller. The horizontal pod autoscaling controller, running within the Kubernetes control plane, periodically adjusts the desired scale of its target (for example, a Deployment) to match observed metrics such as average CPU utilization, average memory utilization, or any other custom metric you specify.
\end{frame}

\begin{frame}
\frametitle{Recursos - RBAC}
    Daemonsets - conceito
\end{frame}

\begin{frame}
\frametitle{Recursos - CRD}
    Daemonsets - conceito
\end{frame}
