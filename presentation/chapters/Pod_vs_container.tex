\section{Pod vs Container}

\begin{frame}
\frametitle{Pods vs containers}
\textbf{"Pods are the smallest deployable units of computing that you can create and manage in Kubernetes."} \cite{K8s}
\begin{itemize}
	\item \uncover<2->{1 Pod : 1 Container}
	\item \uncover<3->{Grupo de 1 ou mais containers}
		\begin{itemize}
			\item \uncover<4->{Compartilham armazenamento e rede}
			\item \uncover<5->{Especificações para rodar o container}
			\item \uncover<6->{Pode conter initContainers}
			\item \uncover<7->{Pode conter containers efêmeros}
		\end{itemize}
\end{itemize}
\end{frame}

\begin{frame}
\frametitle{Containers efêmeros}
\textbf{Ephemeral containers}:
\begin{itemize}
	\item Geralmente é usado para inspecionar (\textit{Debugging}) o estado do Pod:
	\item \uncover<2->{Não possui portas, portanto \textbf{livenessProbe, readinessProbe} não são permitidos}
	\item \uncover<3->{Recursos do pod são imutáveis, portanto não é permitido configurar recursos}
	\item \uncover<4->{Lista completa de permissões: \href{https://kubernetes.io/docs/reference/generated/kubernetes-api/v1.26/\#ephemeralcontainer-v1-core}{link}}
\end{itemize}
\end{frame}

\begin{frame}
\frametitle{Pods estáticos}
\begin{itemize}
	\item Gerenciados diretamente pelo \textit{kubelet} do nó (indepentente do \textit{API server})
	\item \uncover<2->{O \textit{kubelet} verifica o \textit{Pod} e reinicia caso falhe}
	\item \uncover<3->{\textit{kubelet} verifica o diretório \textit{/etc/kubernetes/manifests} por padrão e cria pods caso seja necessário}
	\item \uncover<4->{Não suporta \textit{Containers efêmeros} e não pode referenciar outros objetos da \textit{API (ConfigMap, Secret, etc\dots)}}
	\item \uncover<5->{Onde usar?}
		\begin{itemize}
			\item \uncover<6->{Nos componentes do control plane\footnote{\url{https://anote.dev/static-pods-in-kubernetes/}}}
		\end{itemize}
\end{itemize}
\end{frame}

\begin{frame}
	\frametitle{\textit{LivenessProbe}, \textit{ReadinessProbe} e \textit{StartUpProbe}}

Os 3 são controlados pelo \textit{kubelet}

\uncover<2->{\textbf{LivenessProbe}:}
\begin{itemize}
	\item \uncover<2->{Tentar fazer a aplicação ficar mais disponível removendo \textit{bugs}}
		\begin{itemize}
			\item \uncover<3->{Pode perceber um \textit{deadlock} na aplicação}
		\end{itemize}
\end{itemize}

\uncover<3->{\textbf{ReadinessProbe}:}
\begin{itemize}
	\item \uncover<3->{É usado para saber quando um container está pronto para aceitar tráfico de rede.}
	\begin{itemize}
		\item \uncover<4->{Pode ser usado para controler os pods que servem como \textit{backend}}
		\item \uncover<5->{Se o pod não estiver pronto, ele é removido do \textit{Load Balancer} do serviço}
	\end{itemize}
\end{itemize}
\end{frame}

\begin{frame}
\textbf{StartupProbe}:
\begin{itemize}
	\item É usado para sabe se o container da aplicação começou.
	\begin{itemize}
		\item \uncover<2->{Pode ser configurado para desativar \textit{liveness} e \textit{readness} até o \textit{startup} tenha sucesso.}
		\item \uncover<3->{Dessa forma o \textit{kubelet} não vai matar aplicações lentas antes de elas inicializarem.}
	\end{itemize}
\end{itemize}
\end{frame}
