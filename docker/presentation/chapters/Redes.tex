\section{Redes}

\begin{frame}
	\frametitle{Redes - Bridge}
	\begin{itemize}
		\item \uncover<1->{Usar a interface de rede \textbf{docker0};}
			\begin{itemize}
				\item \uncover<1->{Cada container vai ter um veth (Virtual ethernet interface);}
				\item \uncover<2->{Cada veth vai estar conectado com o docker0;}
			\end{itemize}
		\item \uncover<3->{Todos os containers na rede vão conseguir se comunicar (Ruim);}
		%\item \uncover<4->{Usa o /etc/resolv.conf da máquina local dentro do docker;}
	\end{itemize}
\end{frame}

\begin{frame}
	\frametitle{Redes - User-Defined Bridge}
	\begin{itemize}
		\item \uncover<1->{Igual ao Bridge}
		\item \uncover<2->{Mais controle das conexões entre os containers}
		\item \uncover<3->{Isolados dos containers dentro do bridge padrão}
	\end{itemize}
\end{frame}

\begin{frame}
	\frametitle{Redes - Host}
	\begin{itemize}
		\item \uncover<1->{Compartilha o ip e as portas com a máquina host}
		\item \uncover<2->{É como se fosse uma aplicação rodando no computador}
		\item \uncover<3->{Acaba não ficando isolado}
	\end{itemize}
\end{frame}

\begin{frame}
	\frametitle{Redes - macVLAN}
	\begin{itemize}
		%\item docker network create -d macvlan --subnet 10.7.1.0/24 --gateway 192.168.0.1 -o parent=enp0s3 macVLANname
		%\item docker run -itd --rm --network lan --ip 192.168.0.24 --name teste
		\item \uncover<1->{Feito para aplicações antigas que precisam conectar diretamente na internet física}
			\begin{itemize}
				\item \uncover<1->{Conseguem um enderaço MAC próprio;}
				\item \uncover<2->{Tem seus próprios IPs;}
			\end{itemize}
		\item \uncover<3->{Desvantagens:}
			\begin{itemize}
				\item \uncover<3->{Existe só um endereço MAC;}
					\begin{itemize}
						\item \uncover<3->{Precisa do modo Prosmiscuous (Um endereço físico pode ter múltiplos endereços MAC)}
					\end{itemize}
				\item \uncover<4->{Vão existir o DHCP do roteador e o DHCP do docker;}
			\end{itemize}
	\end{itemize}
\end{frame}

%\begin{frame}
%	\frametitle{Redes - IPVLAN (L2)}
%	\begin{itemize}
%		\item A interface de rede da máquina é compartilhada;
%		\item Allow the host to share its interface network
%		\item docker network create -d ipvlan --subnet 192.168.20.0/24 --gateway 192.168.0.1 -o parent=enp0s3 ipVLANname
%		\item like switching
%	\end{itemize}
%\end{frame}
%
%\begin{frame}
%	\frametitle{Redes - IPVLAN (L3)}
%	\begin{itemize}
%		\item docker network create -d ipvlan --subnet 192.168.20.0/24 -o parent=enp0s3 -o ipvlan\_mode=l3 --subnet 192.168.95.0/24 ipVLANname
%		\item you dont need to specify the gateway, as it will be the host
%		\item The host becomes like a router
%		\item Layer 3
%		\item They connect to the host like he is a router
%		\item (no broadcast and no arp)
%		\item Like a UTM?
%	\end{itemize}
%\end{frame}
