\section{Segurança}

%\begin{frame}
%	\frametitle{Segurança - Thread Modeling - pt 1}
%	\begin{itemize}
%		\item Quais são os possíveis vetores de ataque?
%			\begin{itemize}
%				\item \uncover<1->{Container Escape (System);}
%					\begin{itemize}
%						\item \uncover<1->{O atacante conseguiu abrir uma shell no container do sistema;}
%						\item \uncover<2->{Ele pode ter acesso de um usuário comum ou escalar privilégios;}
%					\end{itemize}
%				\item \uncover<3->{Other Containers via Network:}
%					\begin{itemize}
%						\item \uncover<3->{Está tentando atacar outro container pela rede;}
%					\end{itemize}
%				\item \uncover<4->{Attacking the Orchestration Tool via Network:}
%					\begin{itemize}
%						\item \uncover<4->{Atacar a ferrameta de orquestração;}
%					\end{itemize}
%			\end{itemize}
%	\end{itemize}
%\end{frame}

%\begin{frame}
%	\frametitle{Segurança - Thread Modeling - pt 2}
%	\begin{itemize}
%		\item Quais são os possíveis vetores de ataque?
%			\begin{itemize}
%				\item \uncover<1->{Resource Starvation:}
%					\begin{itemize}
%						\item \uncover<1->{Usar muita CPU, RAM, internet ou armazenamento;}
%						\item \uncover<2->{Acaba afetando outros containers;}
%					\end{itemize}
%				\item \uncover<3->{Host Compromise:}
%					\begin{itemize}
%						\item \uncover<3->{Atacou diretamente o host;}
%					\end{itemize}
%				\item \uncover<4->{Integrity of Images:}
%					\begin{itemize}
%						\item \uncover<4->{Mudou a imagem em algum momento antes do deploy;}
%					\end{itemize}
%			\end{itemize}
%	\end{itemize}
%\end{frame}

\begin{frame}
	\frametitle{OWASP Docker Top 10}
	\begin{itemize}
		\item D01 - Secure User Mapping;
		\item D02 - Patch Management Strategy;
		\item D03 - Network Segmentation and Firewalling;
		\item D04 - Secure Defaults and Hardening;
		\item D05 - Maintain Security Contexts;
		\item D06 - Protect Secrets;
		\item D07 - Resource Protection;
		\item D08 - Container Image Integrity and Origin;
		\item D09 - Follow Immutable Paradigm;
		\item D10 - Logging;
	\end{itemize}
\end{frame}

\begin{frame}[t]
	\frametitle{Segurança - Ferramentas}
	\begin{itemize}
		\item Docker scan: Segurança de Imagens (Snyk)
		\item \href{https://github.com/docker/docker-bench-security}{Docker bench security}
		\item \href{https://www.cisecurity.org/benchmark/docker}{Guideline para usar docker (CIS)}
		\item InSpec (Segurança e Compliance)
		\item Lynis
		\item \href{https://github.com/OWASP/Docker-Security}{OWASP Docker-security}
		\item \href{https://github.com/kost/dockscan}{Dockscan}
		\item \href{https://www.slideshare.net/jlkinsel/a-fun-comparison-of-docker-vulnerability-scanners/}{Comparações de scanners de vulnerabilidade de containers}
		\item \href{https://sysdig.com/blog/20-docker-security-tools/}{20 de scanners de vulnerabilidade de containers}
		\item \href{https://thenewstack.io/draft-vulnerability-scanners/}{Lista de scanners de vulnerabilidade de container}
		\item SELinux (Security Enchantment Linux): Administrar permissões no Linux
			\begin{itemize}
				\item O Linux usa o DAC (Discretionary Access Control);
				\item O \textit{SELinux} permite o MAC (Mandatory Access Control);
				\item Adiciona categorias/rótulos/perfis em todos os objetos contidos no sistema de arquivos;
			\end{itemize}
	\end{itemize}
\end{frame}

%\begin{frame}[t]
%	\frametitle{Videos}
%	\begin{itemize}
%		\item How I Learned Docker Security the Hard Way (So You Don’t Have To): \url{https://www.youtube.com/watch?v=C343TPOpTzU}
%		\item Learn which Container Security checks you should implement into your Software Development Life Cycle: \url{https://www.youtube.com/watch?v=rcrmTHOIz24}
%	\end{itemize}
%\end{frame}
