\section{Entrega}

\begin{frame}
	\frametitle{Entrega - Objetivos}
	\begin{itemize}
	\item Ambientes ``isolados``:
		\begin{itemize}
			\item \uncover<1->{Ambiente de testes $=$ Ambiente de produção;}
			\item \uncover<2->{Entregas de software de forma mais rápida;}
				\begin{itemize}
					\item \uncover<2->{Não é necessário instalar dependências nas máquinas ($\uparrow$ tempo);}
				\end{itemize}
		\end{itemize}
	\end{itemize}
\end{frame}

\begin{frame}
\frametitle{Entrega - CI/CD}
\begin{itemize}
	\item Docker é ótimo para \textit{continuous delivery} e \textit{continuous integration}:
		\begin{itemize}
			\item \uncover<1->{Desenvolvedores fazem mudanças no código localmente;}
			\item \uncover<2->{No ambiente de teste são executados testes manuais e automáticos;}
			\item \uncover<3->{Erros podem ser reproduzidos e arrumados usando docker;}
			\item \uncover<4->{Então a imagem vai para o ambiente de produção;}
		\end{itemize}
\end{itemize}
\end{frame}

\begin{frame}
\frametitle{Entrega - Produção e escalabilidade}
\begin{itemize}
	\item Responsive deployment and scaling
		\begin{itemize}
			\item \uncover<1->{Pode ser executado em:}
				\begin{itemize}
					\item \uncover<1->{Máquinas físicas;}
					\item \uncover<2->{Máquinas virtuais;}
					\item \uncover<3->{Data centers;}
					\item \uncover<4->{Nuvem;}
					\item \uncover<5->{Misto;}
				\end{itemize}
			\item \uncover<6->{Portabilidade $\rightarrow$ escalar os projetos ou retirar aplicações e serviços (Quase tempo real);}
		\end{itemize}
\end{itemize}
\end{frame}

\begin{frame}
\frametitle{Ciclo de vida}
\framesubtitle{Ciclo de vida}
\begin{itemize}
	\item \uncover<1->{Desenvolver: Aplicação e componentes isolados no container;}
	\item \uncover<2->{Execução: Testar e distribuir containers;}
	\item \uncover<3->{Entrega: Ambiente de produção usando container ou serviço orquestrado;}
\end{itemize}
\end{frame}
