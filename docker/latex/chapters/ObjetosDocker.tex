\section{Docker Objects}

\begin{frame}
\frametitle{Docker Objects - Imagem}
\begin{itemize}
	\item É um \textbf{template}. Portanto, só é permitido ler;
	\item \uncover<2->{Contém instruções para criar um container de \textit{Docker};}
	\item \uncover<3->{Uma \textbf{Imagem} é composta por camadas;}
	\item \uncover<4->{Camada inicial $\rightarrow$ Imagem pequena de Linux;}
	\item \uncover<5->{Imagens existentes $\rightarrow$ Customizadas para gerar novas imagens;}
\end{itemize}
\end{frame}

\begin{frame}[t]
\frametitle{Docker Objects - Imagem personalizada}
\begin{itemize}
	\item Para construir a uma imagem customizada, é preciso construir um \textit{Dockerfile}.
	\item \uncover<2->{\textit{Dockerfile}:}
		\begin{itemize}
			\item \uncover<2->{Arquivo com instruções para construir e executar uma imagem de um container;}
			\item \uncover<3->{Cada instrução no arquivo Dockerfile adiciona uma camada nova na imagem;}
		\end{itemize}
\begin{center}

\begin{multicols}{2}
\begin{tikzpicture}
	\coordinate (A) at (0,0);
	\coordinate (B) at (0,0.6);
	\coordinate (C) at (0,1.2);
	\coordinate (D) at (0,1.8);
	\coordinate (E) at (0,2.4);
	\uncover<4->{\node (rect) at (A) [draw,thick,minimum width=3cm,minimum height=0.5cm] {};}
	\uncover<4->{\node at ($(A)-(0,0.2)$)[above] {\tiny FROM python:3};}
	\uncover<5->{\node (rect) at (B) [draw,thick,minimum width=3cm,minimum height=0.5cm] {};}
	\uncover<5->{\node at ($(B)-(0,0.2)$)[above] {\tiny WORKDIR /usr\dots};}
	\uncover<6->{\node (rect) at (C) [draw,thick,minimum width=3cm,minimum height=0.5cm] {};}
	\uncover<6->{\node at ($(C)-(0,0.2)$)[above] {\tiny RUN pip install fla\dots};}
	\uncover<7->{\node (rect) at (D) [draw,thick,minimum width=3cm,minimum height=0.5cm] {};}
	\uncover<7->{\node at ($(D)-(0,0.2)$)[above] {\tiny COPY . .};}
	\uncover<8->{\node (rect) at (E) [draw,thick,minimum width=3cm,minimum height=0.5cm] {};}
	\uncover<8->{\node at ($(E)-(0,0.2)$)[above] {\tiny CMD ["python",\dots};}
\end{tikzpicture}
\columnbreak
\begin{tikzpicture}
	\uncover<9->{\node (rect) at (A) [red,draw,thick,minimum width=3cm,minimum height=0.5cm] {};}
	\uncover<9->{\node at ($(A)-(0,0.2)$)[above] {\tiny FROM python:3};}
	\uncover<10->{\node (rect) at (B) [red,draw,thick,minimum width=3cm,minimum height=0.5cm] {};}
	\uncover<10->{\node at ($(B)-(0,0.2)$)[above] {\tiny WORKDIR /usr\dots};}
	\uncover<11->{\node (rect) at (C) [draw,thick,minimum width=3cm,minimum height=0.5cm] {};}
	\uncover<11->{\node at ($(C)-(0,0.2)$)[above] {\tiny RUN pip install  uvi\dots};}
	\uncover<12->{\node (rect) at (D) [draw,thick,minimum width=3cm,minimum height=0.5cm] {};}
	\uncover<12->{\node at ($(D)-(0,0.2)$)[above] {\tiny COPY . .};}
	\uncover<13->{\node (rect) at (E) [draw,thick,minimum width=3cm,minimum height=0.5cm] {};}
	\uncover<13->{\node at ($(E)-(0,0.2)$)[above] {\tiny  ENTRYPOINT u\dots};}
\end{tikzpicture}
\end{multicols}

\end{center}
\end{itemize}
\end{frame}

\begin{frame}
\frametitle{Docker Objects - Imagem}
\framesubtitle{Imagem}
\begin{itemize}
	\item Novas customizações $\rightarrow$ Novas camadas na Imagem;
	\item \uncover<2->{Na construção de uma imagem, apenas as camadas que sofreram mudanças vão ser reconstruídas;}
	\item \uncover<3->{Imagens leves, pequenas e rápidas (Comparadas com outras tecnologias);}
\end{itemize}
\end{frame}

\begin{frame}[t]
\frametitle{Docker Objects - Containers}
\begin{itemize}
	\item Instância de uma imagem;
		\begin{itemize}
			\item Configurações definidas na inicialização e na imagem;
			\item \uncover<2->{Mudanças no container não são armazenadas;}
		\end{itemize}
	\item \uncover<3->{Com a API do docker é possível:}
		\begin{itemize}
			\item \uncover<3->{Criar um container;}
			\item \uncover<4->{Executar um container;}
			\item \uncover<5->{Parar um container;}
			\item \uncover<6->{Mover um container;}
			\item \uncover<7->{Deletar um container;}
		\end{itemize}
	\item \uncover<8->{Gerenciar:}
		\begin{itemize}
			\item \uncover<8->{Redes;}
			\item \uncover<9->{Armazenamento;}
			\item \uncover<10->{Imagem;}
		\end{itemize}
\end{itemize}
\end{frame}

\begin{frame}[t]
\frametitle{Docker Objects - Tecnologias}
\begin{itemize}
	\item Linguagem \textbf{Go};
	\item \uncover<2->{Muitas funcionalidades do kernel do linux:}
		\begin{itemize}
			\item \uncover<2->{namespaces: Garante que cada um dos containers tenha a capacidade de se isolar em níveis:}
			\begin{itemize}
				\item \uncover<3->{PID: Ter seus próprios PIDs, mas a máquina host pode ver os PIDs do container;}
				\item \uncover<3->{NET: Ter suas próprias interfaces de redes e portas. Responsável pela comunicação de containers;}
				\item \uncover<3->{MNT: Garante que um sistema de arquivos montado não consiga acessar outro sistema de arquivos montado por outro mnt;}
				\item \uncover<3->{IPC: Ter um SystemV IPC isolado, além de uma fila de mensagems POSIX própria;}
				\item \uncover<3->{UTS: Isolamento do hostname, nome do domínio, versão do SO, etc...;}
			\end{itemize}
			\item \uncover<4->{cgroups: Garantir o controle do consumo de processo;}
			\item \uncover<5->{Linux Security Modules: Permissões do que eu sou permitido fazer;}
		\end{itemize}
\end{itemize}
\end{frame}

\begin{frame}[t]
\frametitle{Docker Objects - Container Runtimes}
\begin{itemize}
	\item As ferramentas \textbf{Container Runtimes} são ferramentas que facilitam a criação de containers de forma isolada e segura.
	\item Open container Initiative runtimes:
		\begin{itemize}
			\item Native Runtimes:
			\begin{itemize}
				\item runC: escrito em go e mantido pelo projeto moby do docker;
				\item Railcar: escrito em rust, mas foi abandonado;
				\item Crun: escrito em c, performatico e leve;
			\end{itemize}
		\end{itemize}
	\item Container Runtime Interface
		\begin{itemize}
			\item containerd: um runtime de alto nível desenvolvido no projeto moby do docker, por padrão usa o runC por baixo dos panos;
			\item cri-o: implementação liderada pela Redhat, feita especificamente para o kubernetes;
		\end{itemize}
\end{itemize}
\end{frame}

\begin{frame}
\frametitle{Docker Objects - Resumo}
\begin{itemize}
	\item Daemon (dockerd): espera por chamads API do docker, e gerencia objetos como imagens, containers, redes e volumes;
	\item \uncover<2->{Cliente (docker): interage com o dockerd;}
		\begin{itemize}
			\item \uncover<2->{Comandos como: docker run, o cliente envia comandos para o dockerd;}
			\item \uncover<3->{O cliente docker pode se comunicar com mais de um daemon;}
		\end{itemize}
	\item \uncover<4->{Docker desktop: é uma aplicação fácil de ser instalada nos ambientes Mac, Windows ou Linux.}
		\begin{itemize}
			\item \uncover<5->{O docker desktop contém o daemon (dockerd), o docker client (docker), docker compose, docker content trust, kubernetes e credential helper;}
		\end{itemize}
\end{itemize}
\end{frame}

\begin{frame}
\frametitle{Docker Objects - Resumo}
\begin{itemize}
	\item Docker registries: Guarda as imagens de docker;
		\begin{itemize}
			\item Docker Hub: é um registro público, por padrão o docker procura por imagens no Docker Hub;
			\item \uncover<2->{É possível ter até seu registro próprio;}
			\item \uncover<3->{Os comandos docker pull, docker run e docker push procuram ou enviam as imagens para o registro configurado na máquina que estão rodando;}
		\end{itemize}
	\item \uncover<4->{Docker objects: Imagens, conteiners, redes, volumes, plugins, etc...}
\end{itemize}
\end{frame}
